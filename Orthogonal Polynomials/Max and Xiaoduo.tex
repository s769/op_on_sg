% --------------------------------------------------------------
% This is all preamble stuff that you don't have to worry about.
% Head down to where it says "Start here"
% --------------------------------------------------------------
 
\documentclass[12pt]{article}
 
\usepackage[margin=1in]{geometry} 
\usepackage{amsmath,amsthm,amssymb, graphicx, float, mathtools, epsfig, caption, subcaption, listings, color, booktabs, tcolorbox}
\usepackage{amsmath,amscd, amsthm, mathrsfs, amssymb, esint}
\usepackage[margin=1in]{geometry}
\usepackage[showboxes,overlay]{textpos}
\usepackage{mathtools}
\DeclarePairedDelimiter\abs{\lvert}{\rvert}
\usepackage{xcolor}
\usepackage{verbatim}
\usepackage{multicol}
\usepackage{graphicx}
\usepackage{float}
%\usepackage{times}
\theoremstyle{plain}
   %\newtheorem{theorem}{Theorem}
\usepackage{biblatex}
\addbibresource{ReferencesForPaper.bib}

\theoremstyle{definition}
    \newtheorem{proposition}{Proposition}
   \newtheorem{axiom}{Axiom}
   \newtheorem{result}{Result}
   \newtheorem{lemma}{Lemma}
   \newtheorem{corollary}{Corollary}
   \newtheorem{conjecture}{Conjecture}
   \newtheorem{problem}{Problem}
   \newtheorem{claim}{Claim}
   \newtheorem{definition}{Definition}
   \newtheorem{technique}{Technique}
   \newtheorem{remark}{Remark}
\newtheorem{example}{Example}
\newenvironment{solution}
  {\begin{proof}[Solution]}
  {\end{proof}}
\definecolor{green}{RGB}{34, 139, 34}
 
%\pagenumbering{gobble}
 
\theoremstyle{remark}
\newcommand*{\QEDB}{\hfill\ensuremath{\square}}
%\newtheorem{remark}{Remark}[section]
%\eqref\let\oldref
%\renewcommand{\eqref}[1]{(\oldref{#1})}
\renewcommand{\ss}{\mathbf{s}}
\newcommand{\BB}{\mathcal{B}}
\newcommand{\xx}{\mathbf{x}}
\newcommand{\LL}{\mathcal{L}}
\newcommand{\II}{\mathcal{I}}
\newcommand{\MM}{\mathcal{M}}
\newcommand{\yy}{\mathbf{y}}
\newcommand{\zz}{\mathbf{z}}
\newcommand{\vv}{\mathbf{v}}
\newcommand{\ww}{\mathbf{w}}
\newcommand{\NN}{\mathbb{N}}
\newcommand{\FF}{\mathcal{F}}
\newcommand{\PP}{\mathcal{P}}
\newcommand{\QQQ}{\mathcal{Q}}
\newcommand{\GG}{\mathcal{G}}
\newcommand{\PO}{\mathbb{P}}
\newcommand{\eps}{\epsilon}
\newcommand{\HH}{\mathcal{H}}
\newcommand{\quo}{\bign /}
\newcommand{\EE}{\mathcal{E}}
\newcommand{\HHH}{\overline{\mathcal{H}}}
\newcommand{\ZZ}{\mathbb{Z}}
\newcommand{\QQ}{\mathbb{Q}}
\newcommand{\RR}{\mathbb{R}}
\newcommand{\CC}{\mathbb{C}}
\newcommand{\NNN}{\mathcal{N}}
\newcommand{\CM}{\mathbb{C}_{ {MA}}}
\newcommand{\CMS}{\mathbb{C}_{ {MAS}}}
\newcommand{\sym}{\mathfrak{S}}
\newcommand{\om}{\omega}
\newcommand{\mini}[2]{\text{min}\{#1,#2\}}
\newcommand{\vectorthree}[3]{\begin{bmatrix} #1\\#2\\#3 \end{bmatrix}}
\newcommand{\vectortwo}[2]{\begin{bmatrix} #1\\#2 \end{bmatrix}}
\let\Vec\mathbf
\newcommand{\Cross}[2]{\Vec{#1}\times\Vec{#2}}
\newcommand{\DPartial}[2]{\frac{\partial #1}{\partial #2}}
\newcommand{\Metric}[2]{||#1 - #2||}
\newcommand{\Proj}[2]{\text{Proj}_{\Vec{#1}}(\Vec{#2})}
\newcommand{\Grad}[1]{\nabla #1}
\newcommand{\DDot}[2]{\Vec{#1}\cdot\Vec{#2}}
\newcommand{\Curl}[1]{\text{curl}\,\textbf{#1}}
\newcommand{\Div}[1]{\text{Div}\,\textbf{#1}}
\newcommand{\ps}{\text{ps}_q^1}
\renewcommand{\mod}{\mathop{\rm \ mod}}
\renewcommand{\Im}{{\rm Im}}
\renewcommand{\bar}{\overline}
%\renewcommand{\Re}{{\rm Re}}
\newcommand{\im}{\text{im }}
\renewcommand{\baselinestretch}{1.5}
\usepackage[T1]{fontenc}
\usepackage{bigfoot} % to allow verbatim in footnote
\usepackage[numbered,framed]{matlab-prettifier}
\tcbuselibrary{theorems}
\newtcbtheorem{theorem}{Theorem}{colback=black!5,colframe=black!35!black,fonttitle=\bfseries}{th}




\lstset{
  style              = Matlab-editor,
  basicstyle         = \mlttfamily,
  escapechar         = ",
  mlshowsectionrules = true,
}


\definecolor{mygreen}{RGB}{28,172,0} % color values Red, Green, Blue
\definecolor{mylilas}{RGB}{170,55,241}


\newcommand\cl[1]{\overline{#1}}
\newcommand\bd[1]{\partial #1}
\newcommand{\ir}{\textrm{int}}
 
\newcommand{\N}{\mathbb{N}} %the set of real numbers
\newcommand{\Z}{\mathbb{Z}} %the set of integers
\newcommand{\Q}{\mathbb{Q}} %the set of rational numbers 
\newcommand{\R}{\mathbb{R}} %the set of real numbers
\newcommand{\E}{\mathcal{E}}
\newcommand\p[1]{P\left\{#1\right\}}
\DeclarePairedDelimiter\ceil{\lceil}{\rceil}
\DeclarePairedDelimiter\floor{\lfloor}{\rfloor}
\newcommand\Cov{\textrm{Cov}}
\newcommand\Var{\textrm{Var}}
\newcommand\ip[2]{\langle#1, #2\rangle}

\renewcommand{\vec}[1]{{\mathchoice
                     {\mbox{\boldmath$\displaystyle{#1}$}}
                     {\mbox{\boldmath$\textstyle{#1}$}}
                     {\mbox{\boldmath$\scriptstyle{#1}$}}
                     {\mbox{\boldmath$\scriptscriptstyle{#1}$}}}}
\newcommand{\norm}[1]{\left\| {#1} \right\|_{\scriptscriptstyle 2}}
\newcommand{\mat}[1]{\mathbf{{#1}}}
\newcommand{\grad}{\nabla}
\newcommand{\hessian}{\nabla^2}
\newcommand{\inner}[2]{\left \langle #1, #2\right \rangle}
\newcommand{\lap}{\Delta}
\newcommand{\Sp}{\mathcal{S}}

%\newenvironment{lemma}[2][Lemma]{\begin{trivlist}
%\item[\hskip \labelsep {\bfseries #1}\hskip \labelsep {\bfseries #2.}]}{\end{trivlist}}

\topmargin-.5in
\textwidth6.6in
\textheight9in
\oddsidemargin0in
\def\ds{\displaystyle}
\def\d{\partial}
\def\bH{{\bf H}}
\def\bM{{\bf M}}
\def\bB{{\bf B}}
\def\bm{{\bf m}}
\def\bS{{\bf S}}
\def\ve{{\varepsilon}}
%\def\fs{\scriptsize}
\def\fs{\footnotesize}
\def\reals{\mathbb{R}}
\def\complex{\mathbb{C}}
\def\scriptO{{{\it O}\kern -.42em {\it `}\kern + .20em}}
\setlength\parindent{0pt}

\begin{document}
\title{Sobolev Orthogonal Polynomials on SG}%replace X with the appropriate due date
\author{} %replace with your name 
 
\maketitle
We gather some notations here that are used in literature, and also in our definition.

We define $\alpha_j, \beta_j, \gamma_j, \eta_j$ such that 
$$\alpha_j = P_{j1}(q_1), \quad \beta_j = P_{j2}(q_1), \quad \gamma_j = P_{j3}(q_1), \quad \eta_j = \partial_nP_{j1}(q_1)$$

We also have that $\partial_nP_{j2}(q_1) = -\alpha_j,\ j \geq 1$ and $\partial_nP_{j3}(q_1) = 3\eta_{j+1},\ j \geq 0$. 

The recurrence relations for the constants are as follows \cite{NSTY}

\begin{gather}
    \begin{cases}
    \alpha_j = \frac{4}{5^j - 5}\sum\limits_{l=1}^{j-1}\alpha_{(j-l)}\alpha_l, & j \geq 2\\
    \beta_j = \frac{2}{15(5^j - 1)}\sum\limits_{l=0}^{j-1}(3\cdot 5^{j-l} - 5^{l+1} + 6)\alpha_{(j-l)}\beta_l, & j \geq 1\\
    \gamma_j = 3\alpha_{j+1}, & j \geq 0\\
    \eta_j = \frac{5^j +1}{2}\alpha_j + 2\sum\limits_{l=0}^{j-1}\eta_l\beta_{(j-l)}, & j \geq 1
    \end{cases}
\end{gather}
with initial values $\alpha_0 = 1,\ \alpha_1 = \frac16, \ \beta_0 = -\frac12, \ \eta_0 = 0, \ \partial_nP_{02}(q_1) = -\frac12$.

\newpage

\textbf{List of Inner Products (Type 1):}

\textbf{Easy version of Sobolov Inner Product with only 1st Laplacian:}

\begin{gather}
\inner{P_{ji}}{P_{ki'}}_S = \int_{SG} P_{ji}P_{ki'} d \mu  + \int_{SG} \lap P_{ji}\lap P_{ki'} d \mu\\
= \int_{SG} P_{ji}P_{ki'} d \mu  + \int_{SG} P_{(j-1)i} P_{(k-1)i'} d \mu
\end{gather}

Computation of specific inner products ($\alpha_i'=\frac12$ if $i=0$; otherwise $\alpha_i'=\alpha_i$):

\begin{gather}
    % Begin 1st equation
    \inner{P_{j1}}{P_{k1}}_S = 2\sum\limits_{l=j-m_{*}}^{j}\left(\alpha_{j-l}\eta_{k+l+1}-\alpha_{k+l+1}\eta_{j-l}\right) + 2\sum\limits_{l=j-m_{*}}^{j-1}\left(\alpha_{j-l-1}\eta_{k+l}-\alpha_{k+l}\eta_{j-l-1}\right) \\
    % Begin 2nd equation
    \inner{P_{j2}}{P_{k2}}_S = -2\sum\limits_{l=j-m_{*}}^{j}\left(\beta_{j-l}\alpha_{k+l+1}-\beta_{k+l+1}\alpha_{j-l}'\right) - 2\sum\limits_{l=j-m_{*}}^{j-1}\left(\beta_{j-l-1}\alpha_{k+l}-\beta_{k+l}\alpha_{j-l-1}'\right) \\
    % Begin 3rd equation
    \inner{P_{j3}}{P_{k3}}_S = 18\sum\limits_{l=j-m_{*}}^{j}\left(\alpha_{j-l+1}\eta_{k+l+2}-\alpha_{k+l+2}\eta_{j-l+1}\right)+18 \sum\limits_{l=j-m_{*}}^{j-1}\left(\alpha_{j-l}\eta_{k+l+1}-\alpha_{k+l+1}\eta_{j-l}\right) \\
    % Begin 4th equation
    \inner{P_{j1}}{P_{k2}}_S = -2\sum\limits_{l=j-m_{*}}^{j}\left(\alpha_{j-l}\alpha_{k+l+1}+\beta_{k+l+1}\eta_{j-l}\right)-2
    -2\sum\limits_{l=j-m_{*}}^{j-1}\left(\alpha_{j-l-1}\alpha_{k+l}+\beta_{k+l}\eta_{j-l-1}\right)\\
    \inner{P_{j1}}{P_{k3}}_{S}=\inner{P_{j2}}{P_{k3}}_{S}=0\\
    \inner{P_{j3}^{(n)}}{P_{k3}^{(n)}}_{S}=\inner{P_{j3}^{(0)}}{P_{k3}^{(0)}}_{S}\\
    \inner{P_{j3}^{(n)}}{P_{k3}^{(n')}}_{S}=-\frac{1}{2}\inner{P_{j3}^{(0)}}{P_{k3}^{(0)}}_{S} (n\neq n') \\
    \inner{P_{0i}}{P_{ki'}}_{S}=\inner{P_{0i}}{P_{ki'}}, \inner{P_{ji}}{P_{0i'}}_{S}=\inner{P_{ji}}{P_{0i'}}
\end{gather}
Then for a fixed k = 1,2 or 3, we obtain $\{p_{j}\}_{j=0}^\infty$ and $\{Q_{j}\}_{j=0}^\infty$ by applying Gram-Schmidt process to $\{P_{jk}\}_{j=0}^\infty$, i.e. $p_j=P_{jk}-\sum\limits_{l=1}^{j-1}d_l^2\inner{P_{jk}}{p_l}p_l$, $d_j:=||p_j||^{-1}$ and $Q_j=d_jp_j$ for any $j \geq 0$. \\
For any $0 < r <\infty$, there exists constants $c_1, c_r$ such that 
\begin{gather}
   ||p_j||= d_j^{-1}\leq||P_{jk}||\le c_1((j-1)!)^{-\frac{log5}{log2}}+c_rr^{-j}
\end{gather}
for all $j \geq 0$, which in particular implies that $\lim_{j\to \infty}||p_j||=0$.
\newpage

\textbf{Generalized version of Sobolov Inner Product with up to m-th Laplacian:}

\begin{gather}
    \inner{P_{ji}}{P_{ki'}}_S = \int_{SG} P_{ji}P_{ki'} d \mu  + \sum\limits_{r=1}^{m}\lambda_{r}\int_{SG} \lap^{r} P_{ji}\lap^{r} P_{ki'} d \mu\\
    = \int_{SG} P_{ji}P_{ki'} d \mu  +\sum\limits_{r=1}^{m} \lambda_{r}\int_{SG} P_{(j-r)i} P_{(k-r)i'} d \mu 
    = \sum\limits_{r=0}^{m} \lambda_{r}\int_{SG} P_{(j-r)i} P_{(k-r)i'} d \mu
\end{gather}

Computation of specific inner products:

\begin{gather}
    % Begin 1st equation
    \inner{P_{j1}}{P_{k1}} = 2\sum\limits_{r=0}^{m}\lambda_{r}\sum\limits_{l=j-m_{*}}^{j}\left(\alpha_{j-l-r}\eta_{k+l+1-r}-\alpha_{k+l+1-r}\eta_{j-l-r}\right)\\
    % Begin 2nd equation
    \inner{P_{j2}}{P_{k2}}_S = -2\sum\limits_{r=0}^{m}\lambda_{r}\sum\limits_{l=j-m_{*}}^{j}\left(\beta_{j-l-r}\alpha_{k+l+1-r}-\beta_{k+l+1-r}\alpha_{j-l-r}'\right)\\
    % Begin 3rd equation
    \inner{P_{j3}}{P_{k3}}_S = 18\sum\limits_{r=0}^{m}\lambda_{r}\sum\limits_{l=j-m_{*}}^{j}\left(\alpha_{j-l+1-r}\eta_{k+l+2-r}-\alpha_{k+l+2-r}\eta_{j-l+1-r}\right)\\
    % Begin 4th equation
    \inner{P_{j1}}{P_{k2}}_S = -2\sum\limits_{r=0}^{m}\lambda_{r}\sum\limits_{l=j-m_{*}}^{j}\left(\alpha_{j-l-r}\alpha_{k+l+1-r}+\beta_{k+l+1-r}\eta_{j-l-r}\right)\\
    \inner{P_{j1}}{P_{k3}}_{S}=\inner{P_{j2}}{P_{k3}}_{S}=0\\
    \inner{P_{j3}^{(n)}}{P_{k3}^{(n)}}_{S}=\inner{P_{j3}^{(0)}}{P_{k3}^{(0)}}_{S}\\
    \inner{P_{j3}^{(n)}}{P_{k3}^{(n')}}_{S}=-\frac{1}{2}\inner{P_{j3}^{(0)}}{P_{k3}^{(0)}}_{S}\\
    \inner{P_{0i}}{P_{ki'}}_{S}=\inner{P_{0i}}{P_{ki'}}, \inner{P_{ji}}{P_{0i'}}_{S}=\inner{P_{ji}}{P_{0i'}}
\end{gather}

\textbf{Remark:}
$j\geq m$, $k\geq m$, or $m=\min\{j, k\}$ because $\lap^{r}P_{ji}=0$ if $r>j$.

\newpage

\textbf{List of Inner Products (Type 2):} 
\begin{gather}
\inner{P_{ji}}{P_{ki'}}_H = \int_{SG} P_{ji}P_{ki'} d \mu  + \E(P_{ji}, P_{ki'})\\
= \int_{SG} P_{ji}P_{ki'} d\mu - \int_{SG} (\lap P_{ji}) P_{ki'} d\mu + \sum_{n = 0}^2 P_{ki'}(q_n)\partial_nP_{ji}(q_n)\\
= \int_{SG} P_{ji}P_{ki'} d\mu - \int_{SG} P_{(j-1),i} P_{ki'} d\mu + \sum_{n = 0}^2 P_{ki'}(q_n)\partial_nP_{ji}(q_n)\\
= \inner{P_{ji}}{P_{ki'}} - \inner{P_{(j-1),i}}{P_{ki'}} + \sum_{n = 0}^2 P_{ki'}(q_n)\partial_nP_{ji}(q_n)
\end{gather}

Note that the sum is given by the following for the monomial basis:

\begin{gather*}
    \sum_{n = 0}^2 P_{k1}(q_n)\partial_nP_{j1}(q_n) = 2\alpha_k\eta_j\\
    \sum_{n = 0}^2 P_{k2}(q_n)\partial_nP_{j2}(q_n) = -2\beta_k\alpha_j\\
    \sum_{n = 0}^2 P_{k3}(q_n)\partial_nP_{j3}(q_n) = 6\gamma_k\eta_k\\
    \sum_{n = 0}^2 P_{k2}(q_n)\partial_nP_{j1}(q_n) = 2\beta_k\eta_j\\
    \sum_{n = 0}^2 P_{k3}(q_n)\partial_nP_{j1}(q_n) = 0\\
    \sum_{n = 0}^2 P_{k3}(q_n)\partial_nP_{j2}(q_n) = 0\\
\end{gather*}

using symmetry arguments and observing that the monomials vanish at $q_0$.


The inner products between the monomials are then given by

\begin{align}
    \begin{cases}
    \inner{P_{j1}}{P_{k1}}_H =  &2\alpha_k\eta_j + 2\sum\limits_{l=j-m_*}^j \alpha_{(j-l)}\eta_{(k+l+1)} - \alpha_{(k+l+1)}\eta_{(j-l)} \\[20pt]&- 2\sum\limits_{l=j-m_*-1}^{j-1} \alpha_{(j-l-1)}\eta_{(k+l+1)} - \alpha_{(k+l+1)}\eta_{(j-l-1)} \\[20pt]
    \inner{P_{j2}}{P_{k2}}_H = &+ 2\beta_k\alpha_j -2\sum\limits_{l=j-m_*}^j \beta_{(j-l)}\alpha_{(k+l+1)} - \beta_{(k+l+1)}\alpha_{(j-l)} \\[20pt]&- 2\sum\limits_{l=j-m_*-1}^{j-1} \beta_{(j-l-1)}\alpha_{(k+l+1)} - \beta_{(k+l+1)}\alpha_{(j-l-1)} \\[20pt]
    \inner{P_{j3}}{P_{k3}}_H =  &6\gamma_k\eta_k + 18\sum\limits_{l=j-m_*}^j \alpha_{(j-l+1)}\eta_{(k+l+2)} - \alpha_{(k+l+2)}\eta_{(j-l+1)} \\[20pt]&- 18\sum\limits_{l=j-m_*-1}^{j-1} \alpha_{(j-l)}\eta_{(k+l+2)} - \alpha_{(k+l+2)}\eta_{(j-l)} \\[20pt]
    \inner{P_{j1}}{P_{k2}}_H = &2\beta_k\eta_k -2\sum\limits_{l=0}^j \alpha_{(j-l)}\alpha_{(k+l+1)} + \beta_{(k+l+1)}\eta_{(j-l)} \\[20pt]&+2 \sum\limits_{l=0}^{j-1} \alpha_{(j-l-1)}\alpha_{(k+l+1)} + \beta_{(k+l+1)}\eta_{(j-l-1)} \\[20pt]
    \inner{P_{j1}}{P_{k3}}_H = 0\\
    \inner{P_{j2}}{P_{k3}}_H = 0
    \end{cases}
\end{align}
(Note that if $j = 0$, the second summation is to be taken as 0).

where $m_* = \min(j,k)$ and $\alpha_j, \beta_j, \gamma_j, \eta_j$ are such that 
$$\alpha_j = P_{j1}(q_1), \quad \beta_j = P_{j2}(q_1), \quad \gamma_j = P_{j3}(q1), \quad \eta_j = \partial_nP_{j1}(q_1)$$

We also have that $\partial_nP_{j2}(q_1) = -\alpha_j,\ j \geq 0$ and $\partial_nP_{j3}(q_1) = 3\eta_{j+1},\ j \geq 1$. The recurrence relations for the constants are as follows \cite{NSTY}

\begin{gather}
    \begin{cases}
    \alpha_j = \frac{4}{5^j - 5}\sum\limits_{l=1}^{j-1}\alpha_{(j-l)}\alpha_l, & j \geq 2\\
    \beta_j = \frac{2}{15(5^j - 1)}\sum\limits_{l=0}^{j-1}(3\cdot 5^{j-l} - 5^{l+1} + 6)\alpha_{(j-l)}\beta_l, & j \geq 1\\
    \gamma_j = 3\alpha_{j+1}, & j \geq 1\\
    \eta_j = \frac{5^j +1}{2}\alpha_j + 2\sum\limits_{l=0}^{j-1}\eta_l\beta_{(j-1)}, & j \geq 1
    \end{cases}
\end{gather}
with initial values $\alpha_0 = 1,\ \alpha_1 = \frac16, \ \beta_0 = -\frac12, \ \eta_0 = 0, \ \partial_nP_{02}(q_1) = \frac12$.
\end{document}